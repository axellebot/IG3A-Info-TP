%%%%%%%%%%%%%%%%%%%%%%%%%%%%%%%%%%%%%%%%%%%%%%%%%%%%%%%%%%%%%%%%%%%%%%%%%%%%%%%%
% The Legrand Orange Book
% LaTeX Template
% Version 2.2 (30/3/17)
%
% This template has been downloaded from:
% http://www.LaTeXTemplates.com
%
% Original author:
% Mathias Legrand (legrand.mathias@gmail.com) with modifications by:
% Vel (vel@latextemplates.com)
%
% License:
% CC BY-NC-SA 3.0 (http://creativecommons.org/licenses/by-nc-sa/3.0/)
%
% Compiling this template:
% This template uses biber for its bibliography and makeindex for its index.
% When you first open the template, compile it from the command line with the
% commands below to make sure your LaTeX distribution is configured correctly:
%
% 1) pdflatex main
% 2) makeindex main.idx -s StyleInd.ist
% 3) biber main
% 4) pdflatex main x 2
%
% After this, when you wish to update the bibliography/index use the appropriate
% command above and make sure to compile with pdflatex several times
% afterwards to propagate your changes to the document.
%
% This template also uses a number of packages which may need to be
% updated to the newest versions for the template to compile. It is strongly
% recommended you update your LaTeX distribution if you have any
% compilation errors.
%
% Important note:
% Chapter heading images should have a 2:1 width:height ratio,
% e.g. 920px width and 460px height.
%
%%%%%%%%%%%%%%%%%%%%%%%%%%%%%%%%%%%%%%%%%%%%%%%%%%%%%%%%%%%%%%%%%%%%%%%%%%%%%%%%

%----------------------------------------------------------------------------------------
%	DOCUMENT CONFIGURATIONS
%----------------------------------------------------------------------------------------

\documentclass[12pt,a4paper]{book} % Default font size and left-justified equations

\input{structure} % Insert the commands.tex file which contains the majority of the structure behind the template

\begin{document}

%----------------------------------------------------------------------------------------

%----------------------------------------------------------------------------------------
%	TITLE PAGE
%----------------------------------------------------------------------------------------

\begingroup
\thispagestyle{empty}
\begin{tikzpicture}[remember picture,overlay]
\node[inner sep=0pt] (background) at (current page.center) {\includegraphics[width=\paperwidth]{./Pictures/background.png}};
\draw (current page.center) node [fill=ocre!30!white,fill opacity=0.6,text opacity=1,inner sep=1cm]{\Huge\centering\bfseries\sffamily\parbox[c][][t]{\paperwidth}{\centering Informatique\\[15pt] % Book title
{\Large Compte Rendu de Travaux Pratiques}\\[20pt] % Subtitle
{\huge Axel LE BOT}}}; % Author name
\end{tikzpicture}
\vfill
\endgroup

%----------------------------------------------------------------------------------------
%	COPYRIGHT PAGE
%----------------------------------------------------------------------------------------

\newpage
~\vfill
\thispagestyle{empty}

\noindent Copyright \copyright\ 2017-2018 Axel LE BOT \\ % Copyright notice

\noindent Licensed under the Creative Commons Attribution-NonCommercial 3.0 Unported License (the ``License''). You may not use this file except in compliance with the License. You may obtain a copy of the License at \url{http://creativecommons.org/licenses/by-nc/3.0}. Unless required by applicable law or agreed to in writing, software distributed under the License is distributed on an \textsc{``as is'' basis, without warranties or conditions of any kind}, either express or implied. See the License for the specific language governing permissions and limitations under the License.\\ % License information

%----------------------------------------------------------------------------------------
%	INTRODUCTION
%----------------------------------------------------------------------------------------

\pagestyle{empty} % No headers

\chapterimage{./Pictures/cover-start}

\chapter*{Introduction}\addcontentsline{toc}{part}{\texorpdfstring{\protect\@myparttocformat{Introduction}}{Introduction}}

Ce rapport contient le compte-rendu des travaux pratiques d’Informatique (Module ITC313) concernant les parties UNIX ainsi que C/C++.
Nous allons appliqué les différents principes informatiques étudiés en cours comme des appels systèmes, la gestion des processus et l'héritage.
Nous allons aussi étudier la manipulation de structures de données, la communication distante via l’utilisation de sockets de communication ou encore le principe de l’héritage.
L’objectif de ces travaux pratiques est d’appréhender et d’approfondir ces différentes parties qui sont essentielles dans la compréhension de l’informatique dans sa globalité, ces notions sont des bases nécessaires au développement de programmes informatiques.

\cleardoublepage % Forces the table of contents chapter to start on an odd page so it's on the right

\pagestyle{fancy} % Print headers again

%----------------------------------------------------------------------------------------
%	TABLE OF CONTENTS
%----------------------------------------------------------------------------------------

%\usechapterimagefalse % If you don't want to include a chapter image, use this to toggle images off - it can be enabled later with \usechapterimagetrue

\chapterimage{./Pictures/cover-table_of_contents} % Table of contents heading image

\pagestyle{empty} % No headers

\tableofcontents % Print the table of contents itself

\cleardoublepage % Forces the table of content to start on an odd page so it's on the right

\pagestyle{fancy} % Print headers again

%----------------------------------------------------------------------------------------
%	TABLE OF FIGURES
%----------------------------------------------------------------------------------------

\chapterimage{./Pictures/cover-table_of_contents} % Table of contents heading image

\pagestyle{empty} % No headers

\listoffigures % Print the table of contents itself

\cleardoublepage % Forces the first chapter to start on an odd page so it's on the right

\pagestyle{fancy} % Print headers again

%----------------------------------------------------------------------------------------
%	PART ONE
%----------------------------------------------------------------------------------------

\part{Shell}

\chapterimage{./Pictures/cover-writing} % Chapter heading image
\chapter{TP1+TP2 : Tableaux, matrices et Fonctions recursives}

\section{Exercice sur des tableaux}
\textit{Dans cette partie on verra la différence de complexités de manipulation de tableau entre les tableaux qui sont triés et ceux qui sont non triés.}

\subsection{Fonction sur les tableaux non triés}

\subsubsection{Exercice 1 : Algorithmes de parcours classiques sur tableau non triés}
\textit{Dans cet exercice, nous nous intéressons à des algorithmes de parcours classiques sur les tableaux non triés. Nous allons définir des fonctions de parcours, et organiser le code en plusieurs fichiers. Découper le code en plusieurs fichiers permet non seulement une meilleure lecture, mais améliore la vitesse de compilation.}
\\\\
J'ai écrit une fonction \mintinline{cpp}{sommeElementsTab()} qui prend en paramètres un tableau d’entiers tab et sa taille taille, et renvoie la somme des éléments contenus dans le tableau.
% sommeElementsTab
\inputminted[linenos,firstline=7,lastline=14]{cpp}{../sources/cpp/TP1-2/manipTableauxNonTries.c}

J'ai écrit une fonction \mintinline{cpp}{moyenneValeursTab()} qui prend en paramètres un tableau d’entiers tab et sa taille taille, et renvoie la moyenne des valeurs contenues dans le tableau.
% moyenneValeursTab
\inputminted[linenos,firstline=16,lastline=18]{cpp}{../sources/cpp/TP1-2/manipTableauxNonTries.c}

J'ai écrit une fonction \mintinline{cpp}{valeurContenueDansTabNonTrie()} qui prend en paramètres un tableau d’entiers tab et sa taille taille, ainsi qu’une valeur val, et renvoie 1 si le tableau contient au moins une occurence de la valeur val, et 0 sinon.
% valeurContenueDansTabNonTrie
\inputminted[linenos,firstline=20,lastline=28]{cpp}{../sources/cpp/TP1-2/manipTableauxNonTries.c}

Pour finir cet exercice, j'ai écrit une fonction \mintinline{cpp}{nbOccurencesValeurDansTabNonTrie()} qui prend en paramètres un tableau d’entierstab et sa taille taille, ainsi qu’une valeur val, et renvoie le nombre de fois que la valeur val est contenue dans le tableau.
% nbOccurencesValeurDansTabNonTrie
\inputminted[linenos,firstline=30,lastline=37]{cpp}{../sources/cpp/TP1-2/manipTableauxNonTries.c}

\subsubsection{Exercice 2 : Ajout et suppression d'éléments tableaux non triés}
\textit{Dans cet exercice, nous nous intéressons aux algorithmes modifiant le nombre d’éléments d’un tableau non triés, mais dont l’espace mémoire ne peut varier.}
\\\\
Nous implémenterons ici les fonctions pouvant Être utiles pour modifier des tableaux non triés.\\

J'ai écrit une fonction \mintinline{cpp}{ajoutValeurTabNonTrie()} qui prend en paramètres un tableau d’entier tab, sa taille maximum tailleMax, le nombre réel d’éléments qu’il contient taille, et une valeur à ajouter val. Cette fonction essaye d’ajouter \textbf{val} au tableau tab (par exemple à la fin du tableau), et retourne 1 si l’ajout a pu être correctement réalisé, et 0 sinon.
% ajoutValeurTabNonTrie
\inputminted[linenos,firstline=7,lastline=13]{cpp}{../sources/cpp/TP1-2/modifTableauxNonTries.c}

J'ai écrit une fonction \mintinline{cpp}{supprimeValeurTabNonTrie()} qui prend en paramètres un tableau d’entier tab, le nombre réel d’éléments qu’il contient taille, et une valeur à supprimer val. Cette fonction essaye de supprimer la première occurence de val rencontrée dans le tableau tab, et retourne 1 si la suppression a pu être correctement réalisée (le tableau contenait cette valeur), et 0 sinon. Pour supprimer une telle valeur dans un tableau non trié, une astuce consiste à chercher l’indice de la première cellule contenant la valeur à supprimer, à inverser cette valeur avec la valeur de la dernière cellule, et considérer que la taille du tableau sera diminuée de 1. L’inversion des valeurs de deux cellules n’a pas d’incidence ici puisque le tableau n’est pas censé être trié.
% supprimeValeurTabNonTrie
\inputminted[linenos,firstline=15,lastline=27]{cpp}{../sources/cpp/TP1-2/modifTableauxNonTries.c}

J'ai écrit une fonction \mintinline{cpp}{supprimeToutesLesValeursTabNonTrie()} qui prend en paramètres un tableau d’entier tab, le nombre réel d’éléments qu’il contient taille, et une valeur à supprimer val. Cette fonction essaye de supprimer toutes les occurences de val rencontrée dans le tableau tab, et retourne le nombre d’éléments qui ont été supprimés et 0 sinon.
% supprimeToutesLesValeursTabNonTrie
\inputminted[linenos,firstline=29,lastline=35]{cpp}{../sources/cpp/TP1-2/modifTableauxNonTries.c}

\subsection{Algorithmes de tri de tableaux}
\subsubsection{Exercice 3 : Trier des tableaux aléatoires}
\textit{Dans cet exercice, nous allons écrire en C deux fonctions de tri de tableau, à partir des algorithmes de ces méthodes de tri. Dans cet exercice, nous allons nous baser sur les travaux réalisés dans les exercices précédents, en ajoutant de nouvelles fonctions dans un nouveau fichier, de la même façon que nous l’avons fait dans l’exercice 2}
\\\\
Nous avons défini les fonctions précédentes pour le cas d’un tableau aléatoire, non trié.\\
Il existe plusieurs algorithmes permettant de trier un tableau, ici nous allons utiliser deux de ces algorithmes et les implémenter afin de l'utiliser sur nos tableaux non triés.

J'ai écrit une fonction \mintinline{cpp}{triABulle()} qui prend en paramètre un tableau d’entiers et sa taille (effective), et trie les éléments du tableau selon l’algorithme de tri tri à bulle. Le principe du tri à bulles est de comparer deux valeurs adjacentes et d'inverser leur position si elles sont mal placées.
% triABulle
\inputminted[linenos,firstline=7,lastline=18]{cpp}{../sources/cpp/TP1-2/algosTri.c}

J'ai écrit une fonction \mintinline{cpp}{triSelection()} qui prend en paramètre un tableau d’entiers et sa taille (effective), et trie les éléments du tableau selon l’algorithme de tri tri par sélection. Le principe de cet algorithme est rappelé ci-après. Il vous revient de définir les variables intermédiaires au bon endroit, ou d’adapter les valeurs en fonction du cas de figure.
L'idée est simple : rechercher le plus grand élément (ou le plus petit), le placer en fin de tableau (ou en début), recommencer avec le second plus grand (ou le second plus petit), le placer en avant-dernière position (ou en seconde position) et ainsi de suite jusqu'à avoir parcouru la totalité du tableau.
% triSelection
\inputminted[linenos,firstline=20,lastline=31]{cpp}{../sources/cpp/TP1-2/algosTri.c}

Nous pouvons maintenant trier les tableaux ce qui nous permet de travailler sur les tableaux triés.

\subsection{Fonctions sur les tableaux triés}

\subsubsection{Exercice 4 : Algorithmes de parcours classiques sur tableau triés}
\textit{Dans cet exercice, nous nous intéressons à des algorithmes de parcours classiques sur les tableaux triés. Il s’agit d’une reprise des questions de l’exercice 1, adaptées aux tableaux triés.}
\\\\
J'ai écrit une fonction \mintinline{cpp}{valeurContenueDansTabTrie()} qui prend en paramètres un tableau d’entiers tab et sa taille taille, ainsi qu’une valeur val, et renvoie 1 si le tableau contient au moins une occurence de la valeur val, et 0 sinon. L’algorithme modifié fonctionne par recherche dichotomique, comme vu en cours : on définit les indices min et max de recherche de la valeur dans le tableau (respectivement initialisés à 0 et taille), et on regarde la valeur contenue dans la case d’indice médian ((min + max) /2). Selon que la valeur lu est plus grande, plus petite ou égale, on fait varier la valeur de l’indice min, ou max, ou on conclut.
% valeurContenueDansTabTrie
\inputminted[linenos,firstline=7,lastline=22]{cpp}{../sources/cpp/TP1-2/manipTableauxTries.c}

J'ai écrit une fonction \mintinline{cpp}{nbOccurencesValeurDansTabTrie()} qui prend en paramètres un tableau d’entierstab et sa taille taille, ainsi qu’une valeur val, et renvoie le nombre de fois que la valeur val est contenue dans le tableau. La version optimisée consiste à rechercher une valeur, comme vu précédemment. Si cette valeur existe, on regarde simplement le nombre de cellules à gauche et à droite contenant cette valeur.
% nbOccurencesValeurDansTabTrie
\inputminted[linenos,firstline=24,lastline=48]{cpp}{../sources/cpp/TP1-2/manipTableauxTries.c}

\subsubsection{Exercice 5 : Ajout et suppression d’éléments sur tableaux triés}
\textit{Dans cet ultime exercice, nous nous intéressons aux algorithmes d’ajout et suppression de valeur sur les tableaux triés. Il s’agit d’une reprise des questions de l’exercice 2, adaptées aux tableaux triés.}
\\\\
J'ai écrit une fonction \mintinline{cpp}{ajoutValeurTabTrie()} qui prend en paramètres un tableau d’entier tab trié, sa taille maximum tailleMax, le nombre réel d’éléments qu’il contient taille, et une valeur à ajouter val. Cette fonction essaye d’ajouter val au tableau tab en respectant le tri (par exemple à la fin du tableau), et retourne 1 si l’ajout a pu être correctement réalisé, et 0 sinon. L’algorithme consiste à décaler les valeurs de 1 à droite en partant de la fin, jusqu’à trouver l’endroit où insérer la valeur.
% ajoutValeurTabTrie
\inputminted[linenos,firstline=7,lastline=25]{cpp}{../sources/cpp/TP1-2/modifTableauxTries.c}

J'ai écrit une fonction \mintinline{cpp}{supprimeValeurTabTrie()} qui prend en paramètres un tableau d’entier tab trié, le nombre réel d’éléments qu’il contient taille, et une valeur à supprimer val. Cette fonction essaye de supprimer la première occurence de val rencontrée dans le tableau tab, et retourne 1 si la suppression a pu être correctement réalisée (le tableau contenait cette valeur), et 0 sinon. Pour supprimer une telle valeur dans un tableau trié, une astuce consiste à chercher l’indice de la première cellule contenant la valeur à supprimer (par exemple en utilisant la recherche dichotomique), puis à décaler les valeurs successives de un vers la gauche jusqu’à la fin du tableau.
% supprimeValeurTabTrie
\inputminted[linenos,firstline=27,lastline=46]{cpp}{../sources/cpp/TP1-2/modifTableauxTries.c}

J'ai écrit une fonction \mintinline{cpp}{supprimeToutesLesValeursTabTrie()} qui prend en paramètres un tableau d’entier tab trié, le nombre réel d’éléments qu’il contient taille, et une valeur à supprimer val. Cette fonction essaye de supprimer toutes les occurences de val rencontrée dans le tableau tab, et retourne le nombre d’éléments qui ont été supprimés et 0 sinon. Cet algorithme est le moins facile : il consiste à trouver une occurence au moyen d’une recherche dichotomique, puis se positionner sur celle la plus à gauche, compter le nombre d’occurences de cette valeur (contenues dans les cellules immédiatement suivantes), et décaler les valeurs suivantes du nombre d’occurences calculé.
% supprimeToutesLesValeursTabTrie
\inputminted[linenos,firstline=48,lastline=81]{cpp}{../sources/cpp/TP1-2/modifTableauxTries.c}

\section{Fonctions récursives}
\textit{L’objectif ce cette partie est d’utiliser les fonctions qui s’appelle elle-même appellé "récursives" et dans comprendre leur implémentation.}

\subsection{Exercice 6 : Definition de fonction recursive}
\textit{Dans cet exercice, il nous est demandé de définir des algorithmes récursifs permettant de calculer deux fonctions mathématiques pour lesquelles nous avons déjà définis des algorithmes itératifs précédemment.}
\\\\
Une fonction récursive est une fonction qui s'appelle elle-même. Si dans le corps de la fonction, nous l'utilisons elle-même, alors elle est récursive.\\
L'algorithme suivant permet de calculer la formation de Leibniz :

\begin{itemize}
  \item Par iteration :
  \inputminted[linenos,firstline=7,lastline=14]{cpp}{../sources/cpp/TP1-2/piLeibniz.c}
  \item Par récursivité :
  \inputminted[linenos,firstline=16,lastline=19]{cpp}{../sources/cpp/TP1-2/piLeibniz.c}
\end{itemize}

L'algorithme suivant permet de calculer un produit factorielle :
\begin{itemize}
  \item Par iteration :
  \inputminted[linenos,firstline=7,lastline=14]{cpp}{../sources/cpp/TP1-2/factorielle.c}
  \item Par récursivité :
  \inputminted[linenos,firstline=16,lastline=21]{cpp}{../sources/cpp/TP1-2/factorielle.c}
\end{itemize}

L'algorithme suivant permet de calculer la valeur d'un nombre harmonique :
\begin{itemize}
  \item Par iteration :
  \inputminted[linenos,firstline=7,lastline=14]{cpp}{../sources/cpp/TP1-2/harmonique.c}
  \item Par récursivité :
  \inputminted[linenos,firstline=16,lastline=21]{cpp}{../sources/cpp/TP1-2/harmonique.c}
\end{itemize}

\subsection{Exercice 7 : Algorithme recursif sur matrice}
\textit{Dans cet exercice nous implémenterons un algorithme récursif appliqué sur une matrice. On verra ainsi que la récursivité permet de solutionner beaucoup plus simplement un problème par rapport à un algorithme séquentiel.}
\\\\
Notre algorithme devra affecter la valeur 2 à chacune des cases de la case selectionné.
Si la case est déjà égale à 2, rien ne sera fait.
Les autres cases ne seront pas modifiés.
Pour cela nous appelerons la fonction sur les cases de la même zone de la case selectionné.\\

Afin de rendre notre algorithme réutilisable nous devront passer en paramètre les dimensions de la matrice utilisé avant de passer la matrice en paramètre, ainsi la fonction connaitra les dimensions de la matrice.

Nous avons créé la fonction récursive suivante :

\inputminted[linenos,firstline=8,lastline=25]{cpp}{../sources/cpp/TP1-2/bubbleBreaker.c}


%----------------------------------------------------------------------------------------
%	PART TWO
%----------------------------------------------------------------------------------------

\part{C++}

\chapterimage{./Pictures/cover-writing} % Chapter heading image
\chapter{TP1+TP2 : Tableaux, matrices et Fonctions recursives}

\section{Exercice sur des tableaux}
\textit{Dans cette partie on verra la différence de complexités de manipulation de tableau entre les tableaux qui sont triés et ceux qui sont non triés.}

\subsection{Fonction sur les tableaux non triés}

\subsubsection{Exercice 1 : Algorithmes de parcours classiques sur tableau non triés}
\textit{Dans cet exercice, nous nous intéressons à des algorithmes de parcours classiques sur les tableaux non triés. Nous allons définir des fonctions de parcours, et organiser le code en plusieurs fichiers. Découper le code en plusieurs fichiers permet non seulement une meilleure lecture, mais améliore la vitesse de compilation.}
\\\\
J'ai écrit une fonction \mintinline{cpp}{sommeElementsTab()} qui prend en paramètres un tableau d’entiers tab et sa taille taille, et renvoie la somme des éléments contenus dans le tableau.
% sommeElementsTab
\inputminted[linenos,firstline=7,lastline=14]{cpp}{../sources/cpp/TP1-2/manipTableauxNonTries.c}

J'ai écrit une fonction \mintinline{cpp}{moyenneValeursTab()} qui prend en paramètres un tableau d’entiers tab et sa taille taille, et renvoie la moyenne des valeurs contenues dans le tableau.
% moyenneValeursTab
\inputminted[linenos,firstline=16,lastline=18]{cpp}{../sources/cpp/TP1-2/manipTableauxNonTries.c}

J'ai écrit une fonction \mintinline{cpp}{valeurContenueDansTabNonTrie()} qui prend en paramètres un tableau d’entiers tab et sa taille taille, ainsi qu’une valeur val, et renvoie 1 si le tableau contient au moins une occurence de la valeur val, et 0 sinon.
% valeurContenueDansTabNonTrie
\inputminted[linenos,firstline=20,lastline=28]{cpp}{../sources/cpp/TP1-2/manipTableauxNonTries.c}

Pour finir cet exercice, j'ai écrit une fonction \mintinline{cpp}{nbOccurencesValeurDansTabNonTrie()} qui prend en paramètres un tableau d’entierstab et sa taille taille, ainsi qu’une valeur val, et renvoie le nombre de fois que la valeur val est contenue dans le tableau.
% nbOccurencesValeurDansTabNonTrie
\inputminted[linenos,firstline=30,lastline=37]{cpp}{../sources/cpp/TP1-2/manipTableauxNonTries.c}

\subsubsection{Exercice 2 : Ajout et suppression d'éléments tableaux non triés}
\textit{Dans cet exercice, nous nous intéressons aux algorithmes modifiant le nombre d’éléments d’un tableau non triés, mais dont l’espace mémoire ne peut varier.}
\\\\
Nous implémenterons ici les fonctions pouvant Être utiles pour modifier des tableaux non triés.\\

J'ai écrit une fonction \mintinline{cpp}{ajoutValeurTabNonTrie()} qui prend en paramètres un tableau d’entier tab, sa taille maximum tailleMax, le nombre réel d’éléments qu’il contient taille, et une valeur à ajouter val. Cette fonction essaye d’ajouter \textbf{val} au tableau tab (par exemple à la fin du tableau), et retourne 1 si l’ajout a pu être correctement réalisé, et 0 sinon.
% ajoutValeurTabNonTrie
\inputminted[linenos,firstline=7,lastline=13]{cpp}{../sources/cpp/TP1-2/modifTableauxNonTries.c}

J'ai écrit une fonction \mintinline{cpp}{supprimeValeurTabNonTrie()} qui prend en paramètres un tableau d’entier tab, le nombre réel d’éléments qu’il contient taille, et une valeur à supprimer val. Cette fonction essaye de supprimer la première occurence de val rencontrée dans le tableau tab, et retourne 1 si la suppression a pu être correctement réalisée (le tableau contenait cette valeur), et 0 sinon. Pour supprimer une telle valeur dans un tableau non trié, une astuce consiste à chercher l’indice de la première cellule contenant la valeur à supprimer, à inverser cette valeur avec la valeur de la dernière cellule, et considérer que la taille du tableau sera diminuée de 1. L’inversion des valeurs de deux cellules n’a pas d’incidence ici puisque le tableau n’est pas censé être trié.
% supprimeValeurTabNonTrie
\inputminted[linenos,firstline=15,lastline=27]{cpp}{../sources/cpp/TP1-2/modifTableauxNonTries.c}

J'ai écrit une fonction \mintinline{cpp}{supprimeToutesLesValeursTabNonTrie()} qui prend en paramètres un tableau d’entier tab, le nombre réel d’éléments qu’il contient taille, et une valeur à supprimer val. Cette fonction essaye de supprimer toutes les occurences de val rencontrée dans le tableau tab, et retourne le nombre d’éléments qui ont été supprimés et 0 sinon.
% supprimeToutesLesValeursTabNonTrie
\inputminted[linenos,firstline=29,lastline=35]{cpp}{../sources/cpp/TP1-2/modifTableauxNonTries.c}

\subsection{Algorithmes de tri de tableaux}
\subsubsection{Exercice 3 : Trier des tableaux aléatoires}
\textit{Dans cet exercice, nous allons écrire en C deux fonctions de tri de tableau, à partir des algorithmes de ces méthodes de tri. Dans cet exercice, nous allons nous baser sur les travaux réalisés dans les exercices précédents, en ajoutant de nouvelles fonctions dans un nouveau fichier, de la même façon que nous l’avons fait dans l’exercice 2}
\\\\
Nous avons défini les fonctions précédentes pour le cas d’un tableau aléatoire, non trié.\\
Il existe plusieurs algorithmes permettant de trier un tableau, ici nous allons utiliser deux de ces algorithmes et les implémenter afin de l'utiliser sur nos tableaux non triés.

J'ai écrit une fonction \mintinline{cpp}{triABulle()} qui prend en paramètre un tableau d’entiers et sa taille (effective), et trie les éléments du tableau selon l’algorithme de tri tri à bulle. Le principe du tri à bulles est de comparer deux valeurs adjacentes et d'inverser leur position si elles sont mal placées.
% triABulle
\inputminted[linenos,firstline=7,lastline=18]{cpp}{../sources/cpp/TP1-2/algosTri.c}

J'ai écrit une fonction \mintinline{cpp}{triSelection()} qui prend en paramètre un tableau d’entiers et sa taille (effective), et trie les éléments du tableau selon l’algorithme de tri tri par sélection. Le principe de cet algorithme est rappelé ci-après. Il vous revient de définir les variables intermédiaires au bon endroit, ou d’adapter les valeurs en fonction du cas de figure.
L'idée est simple : rechercher le plus grand élément (ou le plus petit), le placer en fin de tableau (ou en début), recommencer avec le second plus grand (ou le second plus petit), le placer en avant-dernière position (ou en seconde position) et ainsi de suite jusqu'à avoir parcouru la totalité du tableau.
% triSelection
\inputminted[linenos,firstline=20,lastline=31]{cpp}{../sources/cpp/TP1-2/algosTri.c}

Nous pouvons maintenant trier les tableaux ce qui nous permet de travailler sur les tableaux triés.

\subsection{Fonctions sur les tableaux triés}

\subsubsection{Exercice 4 : Algorithmes de parcours classiques sur tableau triés}
\textit{Dans cet exercice, nous nous intéressons à des algorithmes de parcours classiques sur les tableaux triés. Il s’agit d’une reprise des questions de l’exercice 1, adaptées aux tableaux triés.}
\\\\
J'ai écrit une fonction \mintinline{cpp}{valeurContenueDansTabTrie()} qui prend en paramètres un tableau d’entiers tab et sa taille taille, ainsi qu’une valeur val, et renvoie 1 si le tableau contient au moins une occurence de la valeur val, et 0 sinon. L’algorithme modifié fonctionne par recherche dichotomique, comme vu en cours : on définit les indices min et max de recherche de la valeur dans le tableau (respectivement initialisés à 0 et taille), et on regarde la valeur contenue dans la case d’indice médian ((min + max) /2). Selon que la valeur lu est plus grande, plus petite ou égale, on fait varier la valeur de l’indice min, ou max, ou on conclut.
% valeurContenueDansTabTrie
\inputminted[linenos,firstline=7,lastline=22]{cpp}{../sources/cpp/TP1-2/manipTableauxTries.c}

J'ai écrit une fonction \mintinline{cpp}{nbOccurencesValeurDansTabTrie()} qui prend en paramètres un tableau d’entierstab et sa taille taille, ainsi qu’une valeur val, et renvoie le nombre de fois que la valeur val est contenue dans le tableau. La version optimisée consiste à rechercher une valeur, comme vu précédemment. Si cette valeur existe, on regarde simplement le nombre de cellules à gauche et à droite contenant cette valeur.
% nbOccurencesValeurDansTabTrie
\inputminted[linenos,firstline=24,lastline=48]{cpp}{../sources/cpp/TP1-2/manipTableauxTries.c}

\subsubsection{Exercice 5 : Ajout et suppression d’éléments sur tableaux triés}
\textit{Dans cet ultime exercice, nous nous intéressons aux algorithmes d’ajout et suppression de valeur sur les tableaux triés. Il s’agit d’une reprise des questions de l’exercice 2, adaptées aux tableaux triés.}
\\\\
J'ai écrit une fonction \mintinline{cpp}{ajoutValeurTabTrie()} qui prend en paramètres un tableau d’entier tab trié, sa taille maximum tailleMax, le nombre réel d’éléments qu’il contient taille, et une valeur à ajouter val. Cette fonction essaye d’ajouter val au tableau tab en respectant le tri (par exemple à la fin du tableau), et retourne 1 si l’ajout a pu être correctement réalisé, et 0 sinon. L’algorithme consiste à décaler les valeurs de 1 à droite en partant de la fin, jusqu’à trouver l’endroit où insérer la valeur.
% ajoutValeurTabTrie
\inputminted[linenos,firstline=7,lastline=25]{cpp}{../sources/cpp/TP1-2/modifTableauxTries.c}

J'ai écrit une fonction \mintinline{cpp}{supprimeValeurTabTrie()} qui prend en paramètres un tableau d’entier tab trié, le nombre réel d’éléments qu’il contient taille, et une valeur à supprimer val. Cette fonction essaye de supprimer la première occurence de val rencontrée dans le tableau tab, et retourne 1 si la suppression a pu être correctement réalisée (le tableau contenait cette valeur), et 0 sinon. Pour supprimer une telle valeur dans un tableau trié, une astuce consiste à chercher l’indice de la première cellule contenant la valeur à supprimer (par exemple en utilisant la recherche dichotomique), puis à décaler les valeurs successives de un vers la gauche jusqu’à la fin du tableau.
% supprimeValeurTabTrie
\inputminted[linenos,firstline=27,lastline=46]{cpp}{../sources/cpp/TP1-2/modifTableauxTries.c}

J'ai écrit une fonction \mintinline{cpp}{supprimeToutesLesValeursTabTrie()} qui prend en paramètres un tableau d’entier tab trié, le nombre réel d’éléments qu’il contient taille, et une valeur à supprimer val. Cette fonction essaye de supprimer toutes les occurences de val rencontrée dans le tableau tab, et retourne le nombre d’éléments qui ont été supprimés et 0 sinon. Cet algorithme est le moins facile : il consiste à trouver une occurence au moyen d’une recherche dichotomique, puis se positionner sur celle la plus à gauche, compter le nombre d’occurences de cette valeur (contenues dans les cellules immédiatement suivantes), et décaler les valeurs suivantes du nombre d’occurences calculé.
% supprimeToutesLesValeursTabTrie
\inputminted[linenos,firstline=48,lastline=81]{cpp}{../sources/cpp/TP1-2/modifTableauxTries.c}

\section{Fonctions récursives}
\textit{L’objectif ce cette partie est d’utiliser les fonctions qui s’appelle elle-même appellé "récursives" et dans comprendre leur implémentation.}

\subsection{Exercice 6 : Definition de fonction recursive}
\textit{Dans cet exercice, il nous est demandé de définir des algorithmes récursifs permettant de calculer deux fonctions mathématiques pour lesquelles nous avons déjà définis des algorithmes itératifs précédemment.}
\\\\
Une fonction récursive est une fonction qui s'appelle elle-même. Si dans le corps de la fonction, nous l'utilisons elle-même, alors elle est récursive.\\
L'algorithme suivant permet de calculer la formation de Leibniz :

\begin{itemize}
  \item Par iteration :
  \inputminted[linenos,firstline=7,lastline=14]{cpp}{../sources/cpp/TP1-2/piLeibniz.c}
  \item Par récursivité :
  \inputminted[linenos,firstline=16,lastline=19]{cpp}{../sources/cpp/TP1-2/piLeibniz.c}
\end{itemize}

L'algorithme suivant permet de calculer un produit factorielle :
\begin{itemize}
  \item Par iteration :
  \inputminted[linenos,firstline=7,lastline=14]{cpp}{../sources/cpp/TP1-2/factorielle.c}
  \item Par récursivité :
  \inputminted[linenos,firstline=16,lastline=21]{cpp}{../sources/cpp/TP1-2/factorielle.c}
\end{itemize}

L'algorithme suivant permet de calculer la valeur d'un nombre harmonique :
\begin{itemize}
  \item Par iteration :
  \inputminted[linenos,firstline=7,lastline=14]{cpp}{../sources/cpp/TP1-2/harmonique.c}
  \item Par récursivité :
  \inputminted[linenos,firstline=16,lastline=21]{cpp}{../sources/cpp/TP1-2/harmonique.c}
\end{itemize}

\subsection{Exercice 7 : Algorithme recursif sur matrice}
\textit{Dans cet exercice nous implémenterons un algorithme récursif appliqué sur une matrice. On verra ainsi que la récursivité permet de solutionner beaucoup plus simplement un problème par rapport à un algorithme séquentiel.}
\\\\
Notre algorithme devra affecter la valeur 2 à chacune des cases de la case selectionné.
Si la case est déjà égale à 2, rien ne sera fait.
Les autres cases ne seront pas modifiés.
Pour cela nous appelerons la fonction sur les cases de la même zone de la case selectionné.\\

Afin de rendre notre algorithme réutilisable nous devront passer en paramètre les dimensions de la matrice utilisé avant de passer la matrice en paramètre, ainsi la fonction connaitra les dimensions de la matrice.

Nous avons créé la fonction récursive suivante :

\inputminted[linenos,firstline=8,lastline=25]{cpp}{../sources/cpp/TP1-2/bubbleBreaker.c}

\chapterimage{./Pictures/cover-tree} % Chapter heading image
\chapter{TP3+TP4 : Manipulation des arbres}
\textit{Un arbre binaire de recherche (ABR) est une structure de données pour représenter un ensemble ou un tableau associatif dont les clés appartiennent à un ensemble totalement ordonné. Un arbre binaire de recherche permet des opérations rapides pour rechercher une clé, insérer ou supprimer une clé. Un arbre binaire de recherche est un arbre binaire dans lequel chaque nœud possède une clé, telle que chaque nœud du sous-arbre gauche ait une clé inférieure ou égale à celle du nœud considéré, et que chaque nœud du sous-arbre droit possède une clé supérieure ou égale à celle-ci — selon la mise en œuvre de l'ABR, on pourra interdire ou non des clés de valeur égale. Les nœuds que l'on ajoute deviennent des feuilles de l'arbre.}

\begin{figure}[H]
\centering
\includegraphics[width=200pt]{./cpp/Pictures/tp3+tp4-ABR}
\caption{Arbre Binaire de Recherche}
\label{Arbre Binaire de Recherche}
\end{figure}

\section{Algorithmes sur arborescences binaires de recherche (ABR) non équilibrées}

\subsection{Exercice 1 : Mise en place d’ABR et premiers algorithmes}
\textit{Cet exercice permettra de mettre en place les structures d'ABR (Arbres Binaires de Rechercher) et de créer les fonctions de calcules pour les arbres.}

J'ai écrit une fonction \mintinline{cpp}{nouveauNoeud()} qui prend en paramètre une valeur, et réserve la mémoire nécessaire au stockage d’un nouveau nœud dans le tas. Elle affecte alors l’élément valeur avec la valeur passée en paramètre, et fait pointer filsGauche et filsDroit vers null. Enfin, elle retourne un pointeur vers le nœud créé.
% nouveauNoeud
\inputminted[linenos,firstline=9, lastline=24]{cpp}{../sources/cpp/TP3-4/arbresBasiques.c}

J'ai écrit une fonction \mintinline{cpp}{estVide()} qui prend en paramètres un pointeur sur un noeud de l’arbre, et renvoie 1 si le pointeur a comme valeur NULL, et 0 sinon. Cette fonction permet par la suite de déterminer plus facilement si un sous-arbre est vide.
% estVide
\inputminted[linenos,firstline=76, lastline=78]{cpp}{../sources/cpp/TP3-4/arbresBasiques.c}

J'ai écrit une fonction \mintinline{cpp}{estFeuille()} qui prend en paramètres un pointeur sur un noeud de l’arbre, et renvoie 1 si le noeud pointé est une feuille (le noeud existe et ses deux fils pointent sur NULL), et 0 sinon.
% estFeuille
\inputminted[linenos,firstline=80, lastline=84]{cpp}{../sources/cpp/TP3-4/arbresBasiques.c}

J'ai écrit une fonction \mintinline{cpp}{rechercheValeur()} qui prend deux paramètres, le premier étant un pointeur sur la racine de l’arbre, et le second une valeur v à chercher. La fonction renvoie 1 si la valeur v est contenue dans l’arbre, et 0 sinon. Le déroulement de la fonction est le suivant : un pointeur parcours pointe sur le premier élément de l’arbre. Si cet élément est vide (pointeur null), la valeur v n’existe pas, et on renvoie 0. si cet élément existe, on regarde sa valeur. Si cette valeur est la valeur recherchée, on renvoie 1. Sinon, on déplace le pointeur parcours vers le fils gauche ou le fils droit du noeud selon que la valeur recherchée strictement inférieure ou strictement supérieure à la valeur du noeud pointé. Et on recommence.
% rechercheValeur
\inputminted[linenos,firstline=87, lastline=101]{cpp}{../sources/cpp/TP3-4/arbresBasiques.c}

\subsection{Exercice 2 : Algorithmes récursifs sur arborescences}
\textit{Cet exercice permettra de rajouter des algorithmes récursifs pour les arbres.}

J'ai écrit une fonction \mintinline{cpp}{nbNoeuds()} qui prend en paramètre un pointeur noeudCourant vers le premier nœ ud d’un arbre, et renvoie son nombre de noeuds. Pour écrire cette fonction de manière récursive. Le nombre de noeuds d’un arbre vide (c.a.d. noeudCourant est égal à NULL) est égal à 0. Sinon, il est égal à 1 + le nombre de noeuds de son sous-arbre droit + le nombre de noeuds de son sous-arbre gauche.
% nbNoeuds
\inputminted[linenos,firstline=16, lastline=25]{cpp}{../sources/cpp/TP3-4/arbresFctRecursives.c}

J'ai écrit une fonction \mintinline{cpp}{sommeValArbres()} qui prend en paramètre un pointeur noeudCourant vers le premier nœ ud d’un arbre, et renvoie la somme des valeurs des noeuds. Pour écrire cette fonction de manière récursive. La somme des noeuds d’un arbre vide (c.a.d. noeudCourant est égal à NULL) est égal à 0. Sinon, elle est égale à la valeur contenue dans le noeud + la somme des noeuds de son sous-arbre droit + la somme des noeuds de son sous-arbre gauche.
% sommeValArbres
\inputminted[linenos,firstline=26, lastline=38]{cpp}{../sources/cpp/TP3-4/arbresFctRecursives.c}

J'ai écrit une fonction \mintinline{cpp}{hauteur()} qui prend en paramètre un pointeur noeudCourant vers le premier nœ ud d’un arbre, et renvoie sa hauteur. Pour écrire cette fonction de manière récursive. La hauteur d’un arbre vide (c.a.d. noeudCourant est égal à NULL) est égal à 0. Sinon, elle est égale au maximum des profondeurs de chacun de ses sous-arbres + 1
% hauteur
\inputminted[linenos,firstline=40, lastline=50]{cpp}{../sources/cpp/TP3-4/arbresFctRecursives.c}

J'ai écrit une fonction \mintinline{cpp}{detruireArbre()} qui  prend en paramètre un pointeur vers la racine de l’arbre racine, et détruit l’arbre en libérant sa mémoire. Avant de libérer la mémoire d’un noeud, il faut bien évidemment libérer la mémoire de chacun de ses sous-arbres.
% detruireArbre
\inputminted[linenos,firstline=52, lastline=58]{cpp}{../sources/cpp/TP3-4/arbresFctRecursives.c}

\section{Modification et parcours d’ABR non équilibrées}

\subsection{Exercice 3 : Insertion et suppression de valeurs dans une arborescence}
\textit{Cet exercice permettra de rajouter les algorithmes permettant de modifier les arbres (supprimer et ajouter des noeuds).}

J'ai écrit une fonction \mintinline{cpp}{ajouterValeurABR()} qui prend comme paramètres un pointeur racine vers le premier nœ ud d’un arbre ainsi qu’une valeur, et ajoute cette valeur à l’arbre. De plus, elle renvoie un pointeur vers le premier noeud de l’arbre, au cas où ce dernier aurait changé (ce qui est le cas si l’arbre était vide). Rappelons que l’on ajoute un noeud en tant que nouvelle feuille, à l’unique endroit possible pour maintenir le caractère de recherche de l’arbre binaire.
% ajouterValeurABR
\inputminted[linenos,firstline=8, lastline=19]{cpp}{../sources/cpp/TP3-4/modifArbres.c}

J'ai écrit une fonction \mintinline{cpp}{supprimerValeurABR()} qui prend comme paramètres un pointeur racine vers le premier nœud d’un arbre ainsi qu’une valeur, et supprime cette valeur de l’arbre si cette dernière existait. De plus, elle renvoie un pointeur vers le premier noeud de l’arbre, au cas où ce dernier aurait changé (ce qui est le cas si le premier noeud était la valeur à supprimer). Rappelons que la suppression d’une valeur d’un noeud doit maintenir le caractère de recherche de l’arbre binaire.
% supprimerValeurABR
\inputminted[linenos,firstline=21, lastline=123]{cpp}{../sources/cpp/TP3-4/modifArbres.c}

\section{Parcours d’arborescences binaires}

\subsection{Exercice 4 : Parcours sur arbres}
\textit{Cet exercice permet de lister tous les éléments d'un arbre de manière récursive.}

J'ai écrit une fonction \mintinline{cpp}{parcoursProfondeur()} permettant de réaliser un parcours en profondeur, c’est à dire lister les éléments dans l’ordre suivant : (sous-arbre gauche) element (sous-arbre droit).
% parcoursProfondeur
\inputminted[linenos,firstline=8, lastline=21]{cpp}{../sources/cpp/TP3-4/parcoursArbres.c}

\chapter{TP5 + TP6 hiérarchie de processus, signaux}
    \section{Gestion des signaux : envoi et reception}
        \subsection{Exercice 1 : Droits et signaux}
        \subsection{Exercice 2 : capture de signal et traduction en langage C}
        \subsection{Exercice 3 : Capture de signaux et redirections (exercice difficile)}
        \subsection{Exercice 4 : envoi multiples et capture de signal en C}
    \section{Gestion des processus}
        \subsection{Exercice 5 : Processus en premier-plan / Arriere-plan}
        \subsection{Exercice 6 : Duplication et recouvrement de processus}
    \section{Gestion des processus - Suite}
        \subsection{Exercice 7 : Duplication de processus}
        \subsection{Exercice 8 : Creation et destruction de processus}
        \subsection{Exercice 9 : Evaluation du nombre de processus}
        \subsection{Exercice 10 : Conjonctions, Disjonctions, et Duplication}
        \subsection{Exercice 11 : Terminaison normale de processus}

\chapterimage{./Pictures/cover-socket} % Chapter heading image
\chapter{TP7+TP8 : Communication socket}

\section{Communication distante en utilisant l’outil netcat}

\subsection{Exercice 1 : Découverte de la commande nc : netcat}
\textit{L’objectif de cet exercice est de nous familiariser avec une commande puissante appelée \mintinline{shell}{nc}. Cette commande, inspirée de la commande \mintinline{shell}{cat}, permet d’afficher et de recevoir un flux d’octets, non pas à l’écran et depuis le clavier, mais sur ou depuis le réseau, et plus précisément vers un processus distant identifié par une adresse ip un numéro de port et un mode (connecté "TCP" ou non connecté "UDP")}

Afin de lancer un serveur TCP sur le port 3000 nous utilisons la commande \mintinline{shell}{nc -l 3000}. Elle affichera à l'écran tous les messages recu par le serveur.
Afin de lancer un client TCP qui se connectera au serveur TCP nous utilisons la commande \mintinline{shell}{nc localhost 3000}

\subsection{Exercice 2 : Utilisation de la commande nc : netcat pour le transfert de fichier et l’évaluation de la bande passante}
\textit{La commande netcat, souvent réduite au nom nc, est un utilitaire très puissant qui reprend le principe de la commande cat sur un support réseau. Les possibilités de cette commande sont énormes, et permettent de mettre en place très simplement un serveur en mode connecté ou non connecté pour transférer du texte ou un fichier. Cette commande peut également jouer le rôle de client.}

Pour transférer un fichier il faut :
\begin{itemize}
  \item Coté émetteur : utiliser la commande \mintinline{shell}{nc <ip_address>:<port> > <filename>}
  \item Coté récepteur : utiliser la commande \mintinline{shell}{nc <ip_address>:<port> < <filename>}
\end{itemize}

La commande \mintinline{shell}{time} mise avant \mintinline{shell}{nc} nous pouvons avoir les informations de temps de transfert.

\subsection{Exercice 3 : Une histoire de serveurs concurrents ...}
\textit{Dans cet exercice, nous regardons quelles capacités de cohabitation existent entre des serveurs qui voudraient utiliser le même port de communication}

Il est possible de connecter plusieurs serveur tcp.
\begin{itemize}
  \item Le premier hôte se connectera au premier server TCP créé.
  \item Le deuxième hôte se connectera au deuxième server TCP créé.
  \item ainsi de suite.
\end{itemize}
Il en est de même pour la connexion de plusieurs serveurs UDP.

Si nous souhaitons faire un montage avec un serveur tcp et un serveur udp, les clients tcp communiqueraient avec le serveur tcp et les clients udp avec les serveurs udp.
Il s'affichera un message d'erreur sur les serveur TCP \texttt{Address already use}. Cela signifie que si nous nous connectons sur un port avec plusieurs serveurs nous discuterons avec les deux serveurs.

%\subsection{Exercice 4 : Comprendre une requête HTTP}
%\textit{Dans cet exercice, nous regardons une utilisation simple et originale de \mintinline{shell}{netcat} : comprendre ce que votre navigateur internet envoie comme information lorsqu’il effectue une requête sur un site web}

\section{Développement d’un client et d’un serveur en C}
\subsection{Exercice 5 : Mise en place d’une communication en mode non connecte}
\textit{L’objectif de cet exercice est de découvrir les fonctions et structures de base en C permettant une communication en mode non connecté UDP.}

Nous proposerons le code ci-dessous afin d'implementer un client UDP ayant le comportement demandé :
\inputminted[linenos,firstline=31, lastline=87]{cpp}{../sources/cpp/TP7-8/clientUDP.c}

Nous proposerons le code ci-dessous afin d'implementer un serveur UDP ayant le comportement demandé :
\inputminted[linenos,firstline=34, lastline=104]{cpp}{../sources/cpp/TP7-8/serveurUDP.c}

Après avoir lancé le serveur UDP nous testerons la communication entre le serveur UDP et le client UDP à l'aide de la commande \mintinline{bash}{UDClient <IP> <port> <message>} ainsi nous précisons au client l'IP et le port du serveur mais aussi le message que nous souhaitons envoyer.
\mintinline{shell}{0.0.0.0 5000 "tset ysae"}

Le message reviens dans le même ordre \texttt{tset ysae}.

\subsection{Exercice 6 : Création d’une architecture (client UDP) - (relai UDP-TCP) - (serveur TCP)}

\textit{L'objectif de cet exercice est de manipuler les deux mode (connecté et non connecté), en faisant communiquer un serveur tournant en TCP avec un client tournant en UDP. Cette communication n’est pas directement possible et nécessite un processus intermédiaire qui fera le relai entre le client et serveur.}

À l'aide des instructions donné dans le sujet de l'exercice nous pouvons implémenter le code principale d'un serveur TCP étant similaire à celui d'un serveur UDP:
\inputminted[linenos, firstline=34, lastline=104]{cpp}{../sources/cpp/TP7-8/serveurTCP.c}

Nous pouvons ensuite implémenter le le code du relai UDP-TCP :
\inputminted[linenos, firstline=34, lastline=104]{cpp}{../sources/cpp/TP7-8/relaiUDPTCP.c}

Après avoir lancé le serveur TCP ainsi que le relai UDP-TCP nous testerons la communication entre le serveur UDP, le relai UDP TCP et le server TCP à l'aide de la commande \mintinline{bash}{UDPClient <IP> <port> <message>} ainsi nous précisons au client l'IP et le port du serveur mais aussi le message que nous souhaitons envoyer.
\mintinline{shell}{0.0.0.0 5000 "tset ysae"}

Le message reviens dans le même ordre \texttt{easy test}.

\begin{figure}[H]
\centering
\includegraphics[width=300pt]{./cpp/Pictures/tp7+tp8-relay-UDP-TCP}
\caption{Relai UDP-TCP}
\label{Relai UDP-TCP}
\end{figure}

\section{Exercices bonus}
\subsection{Exercice 7 : Résolution de noms}
\textit{Cet exercice à pour objectif de manipuler la fonction \mintinline{cpp}{gethostbyname()}. Cette fonction permet de transformer des noms de domaines en adresse ip, en interrogeant un serveur DNS.}
Les différents appareils connecté à internet communique entre eux grâce aux adresse IP. Afin d'accéder aux adresse IP depuis un nom de domaine, les appareils demandent à un serveur DNS. DNS signifie système de nom de domaine (Domain Name System en anglais).

On peut facilement prendre pour exemple l'accès à l'adresse \texttt{google.fr} et le schématiser de la manière suivante :

\begin{figure}[H]
\centering
\includegraphics[width=300pt]{./cpp/Pictures/tp7+tp8-DNS}
\caption{Résolution DNS}
\label{Résolution DNS}
\end{figure}

La machine demande au serveur DNS l'adresse IP du domaine \texttt{google.fr} qui lui répond \texttt{8.8.8.8}, ainsi la machine pourra se connecter au serveur à l'adresse indiqué.

À l’aide du manuel et des exemples disponibles sur internet, ainsi que de la documentation de la fonction \mintinline{cpp}{gethostbyname()} permettant la translation d’un nom de domaine vers une adresse IP. J'ai créé un programme qui affiche les adresses IP des noms de domaine "www.yahoo.fr", "www.gmail.com" et "www.u-bourgogne.fr".

\inputminted[linenos,firstline=10, lastline=36]{cpp}{../sources/cpp/TP7-8/getHostByName.c}

%\subsection{Exercice 8 : Serveur multi-client en mode connecté}

\chapterimage{./Pictures/cover-cpp} % Chapter heading image
\chapter{TP9+TP10 : Héritage multiple et modélisation}
        \section{Exercice 1 : Organisation d’un jeu de combat au tour par tour}

%\chapterimage{./Pictures/cover-cpp} % Chapter heading image
\chapter{TP11+TP12 : Projet de synthèse}

\section{Realisation d’un jeu de Puissance 4}

%----------------------------------------------------------------------------------------
%	CONCLUSION
%----------------------------------------------------------------------------------------

\cleardoublepage % Forces the conclusion chapter to start on an odd page so it's on the right

\pagestyle{empty}

\chapterimage{./Pictures/cover-end}

\chapter*{Conclusion}\addcontentsline{toc}{part}{\texorpdfstring{\protect\@myparttocformat{Conclusion}}{Conclusion}}

Les travaux pratiques éffectué m'ont permis de revoir déja rencotnre en DUT et de les approfondir. Ainsi j'ai pu en apprendre plus sur des notions que j'avais en parti acquise, et qui font que j'aime l'informatique puisqu'on en découvre toujours plus.

\pagestyle{fancy}

%----------------------------------------------------------------------------------------

\end{document}
